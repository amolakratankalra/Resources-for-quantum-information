\documentclass{article}
\usepackage[utf8]{inputenc}
\usepackage{amsmath}
\usepackage{braket}
\usepackage{amssymb,amsmath,amsthm}
\newtheorem{theorem}{Theorem}

\title{Pauli \& Clifford Group for qudits}
\author{Amolak Ratan Kalra}
\date{August 2020}

\begin{document}

\maketitle
\tableofcontents
\section{Qudit Pauli Group}
For qubits the Pauli $X$ and $Z$ matrices are the following:
\[X=
\begin{pmatrix}
0 & 1\\
1 & 0\\
\end{pmatrix}~~~
Z=
\begin{pmatrix}
1 & 0\\
0 & -1\\
\end{pmatrix}
\]
The Y matrix is defined to be the product of X and Z matrices with a phase of $i$ also multiplied 
\[
XZ=iY
\]
So these are the Pauli X, Y and Z matrices, these matrices along with the identity matrix form the Pauli group for qubits. They also form a basis.
Note that we can generate the Y matrix upto a phase using the product of X and Z matrices.
We now want to generalize the Pauli group to qudits. The $X$ gates in $d$-dimensions can be defined in the following way:
\[
X=\sum_{k}\ket{k+1}\bra{k}
\]
We can similarly define the $Z$ matrix as the following:
\[
\begin{pmatrix}
1 & 0 & 0 &... & 0\\
0 & \omega & 0 &... & 0\\
0 & 0 & \omega^{2} &... & 0\\
. &...\\
. &...\\
0 & 0 & 0 &... & \omega^{d}\\
\end{pmatrix}
\]
\[
\omega=e^{\frac{2\pi{i}}{d}}
\]
These matrices also satisfy the following commutator relationships
\[
XZ-\omega{ZX}=O_{d}
\]
Here $0_{d}$ is $d\times{d}$ zero matrix. They also satisfy the following property:
\[
X^{d}=Z^{d}=I
\]
Generalized Pauli matrices can be given by the following formula:
\[
U_{ij}=X^{i}Z^{j}
\]
So for example in the two dimensional case we will get the following structure for the Pauli group:
\[
d=2 \implies \omega=e^\frac{2\pi{i}}{2}=e^{i\pi}=-1
\]
So using the formulas above we get
\[
X=
\begin{pmatrix}
0 & 1\\
1 & 0 \\
\end{pmatrix}
\]
which is exactly what we had for the qubit case. For the Pauli Z in 2-dimensions we have:
\[
Z=
\begin{pmatrix}
1 & 0\\
0 & \omega\\
\end{pmatrix}
=
\begin{pmatrix}
1 & 0\\
0 & -1\\
\end{pmatrix}
\]
Using the definition of $\omega$ given above.
For the $3$ dimensional case similarly we can also get the following matrices for $X$ and $Z$, here since $d=3$ we have $\omega=e^\frac{2\pi{i}}{3}$ which are given below:
\[
X=
\begin{pmatrix}
0 & 0 & 1\\
1 & 0 & 0\\
0 & 1 & 0\\
\end{pmatrix}~~~
Z= 
\begin{pmatrix}
1 & 0 & 0\\
0 & \omega & 0\\
0 & 0 & \omega^{2}\\
\end{pmatrix}
\]
\section{Clifford Group}
The question we are interested in is what U satisfy: $UPU^{\dagger} \in P_{n}$ where $P_{n}$ is the Pauli group on n qubits.
So one way to define a Clifford group is all $U$'s which satisfy the above constraint.
\begin{equation}
Cl_{n}=\{U: UP^{\dagger}U \in P_{n} \forall P \in P_{n}\}
\end{equation}
 For qubits the Clifford group consists of: \{H, CNOT, T\} where 
 \[
 T=
 \begin{pmatrix}
 1 & 0\\
 0 & i\\
 \end{pmatrix}
 \]
 In general the $T$ gate for $p \geq 5$ is defined in the following way:
 \[
 T=\sum_{k}\omega^{k^{3}6^{-1}}\ket{k}\bra{k}
 \]
The Pauli operators in p dimensions can also be expressed as:
\[
X\ket{k}=\ket{k+1}
\]
The addition here is mod p.
The Z gate takes the following form:
\[
Z\ket{k}=\omega^{k}\ket{k}
\]
Now for the Clifford product for $p>2$ is generated by $H$ and $S$ these are generalizations of $H$ and $S$ gate for qubits:
\[
S=\sum_{j=0}^{p-1}\omega^{(j)(j+1){2^{-1}}}\ket{j}\bra{j}
\]
\[
H=\frac{1}{\sqrt{p}}\sum_{j=0}^{p-1}\sum_{k=0}^{p-1}\omega^{jk}\ket{j}\bra{k}
\]
Now for a qutrit $p=3$ so we have 
\[
S=\ket{0}\bra{0}+\omega\ket{1}\bra{1}+\omega^{3}\ket{2}\bra{2}
\]
Now note $\omega^{3}=1$ so the above expression becomes:
\[
S=\ket{0}\bra{0}+\omega\ket{1}\bra{1}+\ket{2}\bra{2}
\]
To put this in matrix we note that $\ket{0} \to \ket{0}, \ket{1} \to \omega\ket{1}, \ket{2} \to \ket{2}$
So the following matrix represents such a transformation:
\[
S=
\begin{pmatrix}
1 & 0 & 0\\
0 & \omega & 0\\
0 & 0 & 1 \\
\end{pmatrix}
\]
Now for the expression for $H$ we can follow a similar procedure and expand the summation since the summation is over two different indices there will be cross terms in the matrix expression for $H$.
\[
H=
\begin{pmatrix}
1 & 1 & 1\\
1 & \omega & \omega^{2}\\
1 & \omega^{2} & \omega\\
\end{pmatrix}
\]
Clifford group can be related to elements of SL(2,$Z_{p}$). First the elements of SL(2,$Z_{p}$) are given by matrices of the following form:
\begin{equation}
\begin{pmatrix}
a & b\\
c & d\\
\end{pmatrix}
\end{equation}
Here the determinant of this matrix is 1. so we have $ad-bc=1$
Now usually this would have $p^{3}$ elements but the condition that the determinant must be equal to 1 reduces the total number of elements to $p(p^2-1)$, these elements can be generated by $\hat{S}$ and $\hat{H}$
\[\hat{S}=
\begin{pmatrix}
1 & 0\\
1 & 1\\
\end{pmatrix}~~~
\hat{H}=
\begin{pmatrix}
0 & -1\\
1 & 0\\
\end{pmatrix}
\]
Any element of SL(2, $Z_{p}$) with $b\neq 0$ can be expressed as a product of powers of $\hat{S}$ and $\hat{H}$ in the following way:
\begin{equation*}
E_{z_{p}}=\hat{H}\hat{S}^{m}\hat{H}\hat{S}^{n}\hat{H}\hat{S}^{q}
\end{equation*}
where $E_{z_{p}}$ is any element of $SL(2, Z_{p})$
we can see this in the following way:
\[
\hat{S}^{2}=\hat{S}\hat{S}=
\begin{pmatrix}
1 & 0\\
1 & 1\\
\end{pmatrix}
\begin{pmatrix}
1 & 0\\
1 & 1\\
\end{pmatrix}
=
\begin{pmatrix}
1 & 0\\
2 & 1\\
\end{pmatrix}
\]
Similary $\hat{S}^{3}$ is: $
\begin{pmatrix}
1 & 0\\
3 & 1\\
\end{pmatrix}
$
So a general $\hat{S}^{m}$ can be written as:
\[
\hat{S}^{m}=
\begin{pmatrix}
1 & 0\\
m & 0\\
\end{pmatrix}
\]
So in general for example when we multiply $\hat{H}\hat{S}^{m}$ we get the following 
\[\hat{H}\hat{S}^{m}=
\begin{pmatrix}
0 & -1\\
1 & 0\\
\end{pmatrix}
\begin{pmatrix}
1 & 0\\
m & 1\\
\end{pmatrix}
=
\begin{pmatrix}
-m & -1\\
1 & 0\\
\end{pmatrix}
\]
Now similarly we will have 
\[
\hat{H}\hat{S}^{n}= 
\begin{pmatrix}
-n & -1\\
1 & 0\\
\end{pmatrix}
\]
and $\hat{S^{q}}$ is 
\[
\hat{H}\hat{S^{q}}=
\begin{pmatrix}
-q & -1\\
1 & 0\\
\end{pmatrix}
\]
So therefore we have 
\[
\hat{H}\hat{S}^{m}\hat{H}\hat{S}^{n}\hat{H}\hat{S}^{q}=
\begin{pmatrix}
-m & -1\\
1 & 0\\
\end{pmatrix}
\begin{pmatrix}
-n & -1\\
1 & 0\\
\end{pmatrix}
\begin{pmatrix}
-q & -1\\
1 & 0\\
\end{pmatrix}
=
\begin{pmatrix}
q-qmn+m & n\\
nq & n\\
\end{pmatrix}
\]

\section{Isomorphism between Clifford group}
We need to prove:
\begin{equation}
 D_{\vec{\chi}} V_F= V_F D_{F {\vec{\chi}}}
\end{equation}
and 
\begin{equation}
    V_{F_1} V_{F_2}=V_{F_1 F_2},
\end{equation}
up to overall phases.

We will need the following identities:
\begin{eqnarray}
\sum_{k=0}^{p-1} \omega^k & = & 0 \\
\sum_{k=0}^{p-1} \omega^{k^2} &= & e^{i \phi_p} \sqrt{p} 
\end{eqnarray}
The second relation is a Gauss sum. The phase depends on $p$ in a number theoretic way, but since we are not concerned with overall phases, we can ignore it. 
To check if P is a normal subgroup of C we will do the following calculation:
\[
F^{-1}SF=
\begin{pmatrix}
d & -b\\
-c & a\\
\end{pmatrix}
\begin{pmatrix}
1 & 0\\
n & -1\\
\end{pmatrix}
\begin{pmatrix}
a & b\\
c & d\\
\end{pmatrix}
\]
The result of this calculation is not an element of P therefore $\mathbf{P}$ is not a normal subgroup of $\mathbf{C}$
for $p=3$ we have
\[
\begin{pmatrix}
a & b\\
c & d\\
\end{pmatrix}
\begin{pmatrix}
1 & 0\\
n & 1\\
\end{pmatrix}
=
\begin{pmatrix}
a+bn & b\\
c+dn & d\\
\end{pmatrix}
\]
Now we will have two cases $b=0$ and $b\neq{0}$
if $b=0 \implies d\neq 0$ and $n=-c/d$
The the above expression becomes
\[
\begin{pmatrix}
a & 0\\
0 & a^{-1}\\
\end{pmatrix}
\]
for $b\neq 0$ we get
\[
\begin{pmatrix}
0 & b\\
c^{`} & d\\
\end{pmatrix}
\]


\section{Non-Clifford Qudit Gates}

We can take the operator $T = \sum_{j=0}^{p-1} \omega^{k^36^{-1}} \ket{k}\bra{k}$, as the canonical non-Clifford operator. This operator is in the third level of the Clifford hierarchy.

Other non-Clifford gates we may consider are:
\begin{equation}
    \begin{pmatrix}
    1 & 0 & 0 \\
    0 & 1 & 0 \\
    0 & 0 & -1 
    \end{pmatrix}
\end{equation}
which can arise in state-injection via magic states for qutrits. This is not in any level of the Clifford hierarchy. It may have natural generalizations to qudits.

We can also consider gates in higher levels of the Clifford Hierarchy. These were determined by Gottesman and ...\\

$\mathbf{P}$ is the largest subgroup which satisfies $\mathbf{PT}=\mathbf{TP}$
\[
XT=\sum_{k}\ket{k+1}\bra{k}\sum_{k}\omega^{k^{3}6^{-1}}\ket{k}\bra{k}
\]
\[
XT=\sum_{k}\omega^{k^{3}6^{-1}}\ket{k+1}\bra{k}\ket{k}\bra{k}
\]
Now we know that $\sum_{k}\bra{k}\ket{k}=I$ therefore we get the following expression
\[
\sum_{k}\omega^{k^{3}6^{-1}}\ket{k+1}\bra{k}
\]
We now shift the indices from $k$ to $k-1$ to get the following expression
\[
\sum_{k}\omega^{(k-1)^{3}6^{-1}}\ket{k}\bra{k-1}
\]
We now expand the power of $\omega$ the idea here is to express $XT$ in terms of $S$, $Z$ and $X$. We therefore get the following expression:
\[
\sum_{k}\omega^{(k^{3}-1-3k(k-1))6^{-1}}\ket{k}\bra{k-1}
\]
We can further simplify this expression to get:
\[
\sum_{k}\omega^{k^{3}}6^{-1}\omega^{-2^{-1}k^{2}}\omega^{-2^{-1}k}\omega^{-6^{-1}}\ket{k}\bra{k-1}
\]
We can write the above in the following way:
\begin{equation}
XT=TS^{-1}ZX\omega^{-6^{-1}}
\end{equation}
Now we can also calculate
\[
T^{\dagger}XT=S^{-1}ZX\omega^{-6^{-1}}
\]
\[
T^{\dagger}ZT=Z
\]
We now want to calculate the following:
\[
T^{\dagger}D_{(x,y)}T=T^{\dagger}\omega^{2^{-1xz}}X^{x}Z^{z}T
\]
\[=
\omega^{2^{-1xz}}T^{\dagger}X^{x}TT^{\dagger}Z^{z}T=\omega^{2^{-1xz}}(S^{-1}ZX\omega^{6^{-1}})^{x}Z^{z}
\]
\[
B_{uv}=tr(A_{uv}\sum_{x,z}\omega^{2^{-1xz}}(S^{-1}ZX\omega^{6^{-1}})^{x}Z^{z})
\]
\section{Discrete Wigner Function}
THe Heisenberg-Weyl displacement operators are defined as, \begin{equation}
 D_{(u,v)} = \omega^{uv2^{-1}}X^uZ^v.
\end{equation}
The phase ensures that $D_{\chi_1}D_{\chi_2} = D_{\chi_1+\chi_2}$, and $D_\chi^\dagger = D_{-\chi}$.
The Heisenberg-Weyl displacement operators are unitary but not Hermitian. A manifestly Hermitian basis for single-qudit density matrices is formed by the \textit{discrete phase point operators} $A_{(q,p)}$, which are defined in terms of the Heisenberg-Weyl displacement operators as follows: 
\begin{eqnarray}
A_{(0,0)}&=&\frac{1}{d}\sum_{u=0}^{d-1}\sum_{v=0}^{d-1}  D_{(u,v)} \\ A_{(u,v)}&=&  D_{(u,v)}A_{(0,0)} D_{(u,v)}^\dagger. \label{phase-point-def}
\end{eqnarray}
Any qudit density matrix can be expressed as a linear combination of phase-point operators with real, but possibly negative, coefficients, 
\begin{equation}
W_{(u,v)}=\frac{1}{d} \text{Tr}(\hat\rho A_{(u,v)}).
\end{equation}
This representation of the quantum state $\rho$ is known as its \textit{discrete Wigner function}.

Note that (check this)
\begin{equation}
    D_{\vec \chi} V_F A_{\vec{\eta}} (D_{\vec \chi} V_F)^\dagger
    = A_{F \vec \eta+ \vec \chi}.
\end{equation}

Also,
\begin{equation}
    T
    ^\dagger A_{(0,0)} T
    = ?.
\end{equation}
\section{Last Theorem about $\epsilon$ approximation}
Volume of SU(p) is given by the following formula\cite{p1}:
\[
Vol(SU(p))=\sqrt{p2^{p-1}}\pi^{\frac{(p-1)(p-2)}{2}}\prod_{k=1}^{p-1}\frac{1}{k!}
\]
The total number of operators in canonical form with T-count $m\geq 1$ is 
\[
p((p-1)p)^{m+1}(1+p)^{2}
\]
The total number of forms with T count $\leq n$ is
\[
\frac{p^{3}(p-p^{3}+((p-1)p)^{n}(p^{2}-1)^{2})
}{-1+p(p-1)}
\]
The volume of a $\epsilon$ ball in $p$-dimensions is given by
\[
V=\frac{\pi^{p/2}}{\Gamma(p/2+1)}R^{p}
\]
So the theorem for $\epsilon$ approximation becomes:
\begin{theorem}
Let $\epsilon \leq 0$. There exists $U \in SU(p)$ whose $\epsilon$ approximation by a Clifford+ T circuit requires a number n of $T$ gates where $n\leq \frac{1}{2\log_{e}(p^{2}-p)}(log_{e}\frac{B}{A}+{(p^{2}-1)}\log_{e}\frac{1}{\epsilon})$, where $A=\frac{\pi^{(p^2-1)/2}}{\Gamma((p^2-1)/2)}, B=\sqrt{p2^{p-1}}\pi^{\frac{(p-1)(p-2)}{2}}\prod_{k=1}^{p-1}\frac{1}{k!}$
\end{theorem}
\begin{proof}
The proof is based on the volume counting method. There are a total of $\frac{p^{3}(p-p^{3}+((p-1)p)^{n}(p^{2}-1)^{2})
}{-1+p(p-1)}$ canonical forms of $T$-count $\leq$ n. Each ball of $\epsilon$ radius in $p$-dimensions has the volume $V=\frac{\pi^{p/2}}{\Gamma(p/2+1)}{\epsilon}^{p}
$ as $\epsilon \to 0$. We will use this formula for a $p^{2}-1$ dimensional ball for radius $\epsilon$. We need to cover the full volume of $SU(p)$ so that every operator is $\epsilon$ approximated. Therefore the inequality in $n$ that we have is 
\[
\frac{p^{3}(p-p^{3}+((p-1)p)^{n}(p^{2}-1)^{2})
}{-1+p(p-1)}\frac{\pi^{(p^2-1)/2}}{\Gamma((p^2-1)/2)}{\epsilon}^{p^{2}-1} \leq \sqrt{p2^{p-1}}\pi^{\frac{(p-1)(p-2)}{2}}\prod_{k=1}^{p-1}\frac{1}{k!}
\]
Now we have the following equation
\[
\frac{p^{3}(p-p^{3}+((p-1)p)^{n}(p^{2}-1)^{2})
}{-1+p(p-1)}\frac{\pi^{(p^2-1)/2}}{\Gamma((p^2-1)/2)}{\epsilon}^{p^{2}-1} \leq \sqrt{p2^{p-1}}\pi^{\frac{(p-1)(p-2)}{2}}\prod_{k=1}^{p-1}\frac{1}{k!}
\]
for large $n$ the expression becomes:
\[
\frac{p^{3}((p(p-1))^{n})}{-1+p(p-1)}\frac{\pi^{(p^2-1)/2}}{\Gamma((p^2-1)/2)}{\epsilon}^{p^{2}-1} \leq \sqrt{p2^{p-1}}\pi^{\frac{(p-1)(p-2)}{2}}\prod_{k=1}^{p-1}\frac{1}{k!}
\]
Now we let $A=\frac{p^{3}}{{-1+p(p-1)}}\frac{\pi^{(p^2-1)/2}}{\Gamma((p^2-1)/2)}$ and $B=\sqrt{p2^{p-1}}\pi^{\frac{(p-1)(p-2)}{2}}\prod_{k=1}^{p-1}\frac{1}{k!}$ So we now have the following expression
\[
A((p(p-1))^{n}){\epsilon}^{p^{2}-1}\leq B
\]
Taking log to the base e on both sides we get
\[
n\leq \frac{1}{\log(p(p-1))}(\log(\frac{B}{A}+(p^{2}-1)\log(\frac{1}{\epsilon}))
\]
\end{proof}

\section{The ZX-calculus is complete for the single-qubit Clifford+T
group}
\begin{itemize}
\item ZX calculus is a graphical calculus for pure state qubit quantum mechanics. ZX calculus can be used to express any operation in pure state qubit quantum mechanics that is it is universal. Any equality derived in ZX calculus can be derived using matrix mechanics this is called soundness.
\item The question about completeness it the following: Can any equality derived in matrix mechanics be also derived using rewrite rules of zx calculus? The answer is NO in general but Yes in the some restricted cases like stabilzier quantum mechanics for qubits. 
\item Qubit stabilizer quantum mechanics is characterised by the fact that all allowed states are eigenstates of the tensor product of Pauli operators/matrices with global phase. The stabilizer unitaries are those unitaries which map tensor product of Pauli matrices to to tensor product of Pauli matrices upto a global phase. They form a group called the Clifford group.
\item The Clifford group is generated by $\{H,S, \textrm{CNOT}\}$ now if we add $T$ to this set of operators the set becomes universal for quantum computation. For a single qubit we can ignore the CNOT gate also we have $S=T^{2}$ so the generators become $\{H,T\}$, Just to review the matrix form of these operators is given below:
\[
H=\frac{1}{\sqrt{2}}\begin{bmatrix}
1 & 1\\
1 & -1\\
\end{bmatrix}~~~~S=\begin{bmatrix}
1 & 0\\
0 & i\\
\end{bmatrix}~~~~T=\begin{bmatrix}
1 & 0\\
0 & e^{i\frac{\pi}{4}}\\
\end{bmatrix}
\]
From this one can easily see that $S=T^{2}$
\item What the paper has done is basically the following:
\begin{enumerate}
\item They have used the Matsumoto Amano normal form for the completeness proof. What they show is that an arbitrary single-qubit Clifford+$T$ operator can be brought into the normal form proposed by Matusomoto and then they show its unique. This basically now means that any equality that is derived using Clifford+$T$ operators in matrix mechanics can be derived using ZX calculus rewrite rules.
\end{enumerate}
What we will have to do is do the same proof but for qudits and using a qudit version of ZX calculus.
\item The $W$ gate used in the paper is defined in Matsumoto Amano's original paper and it is:
\[
W_{n}\in \{I, H, PH\}~~~~W_{i}=\{H,PH\}~~~\forall~i=1...n-1
\]
\item The goal here is now to define what $\mathcal{H},~~\mathcal{H'},~~ \mathcal{P}$ look in the diagrammatic notation. Follow Miriam's overall but translate into our paper's notation.
\end{itemize}
\begin{thebibliography}{}
\bibitem{p1} Volumes of Compact Manifolds. Boya, Luis J; Sudarshan, E. C. G.; Tilma, Todd. arXiv:math-ph/0210033 



\end{thebibliography}
\end{document}
